%% Generated by Sphinx.
\def\sphinxdocclass{report}
\documentclass[letterpaper,10pt,english]{sphinxmanual}
\ifdefined\pdfpxdimen
   \let\sphinxpxdimen\pdfpxdimen\else\newdimen\sphinxpxdimen
\fi \sphinxpxdimen=.75bp\relax

\usepackage[utf8]{inputenc}
\ifdefined\DeclareUnicodeCharacter
 \ifdefined\DeclareUnicodeCharacterAsOptional\else
  \DeclareUnicodeCharacter{00A0}{\nobreakspace}
\fi\fi
\usepackage{cmap}
\usepackage[T1]{fontenc}
\usepackage{amsmath,amssymb,amstext}
\usepackage{babel}
\usepackage{times}
\usepackage[Bjarne]{fncychap}
\usepackage{longtable}
\usepackage{sphinx}

\usepackage{geometry}
\usepackage{multirow}
\usepackage{eqparbox}

% Include hyperref last.
\usepackage{hyperref}
% Fix anchor placement for figures with captions.
\usepackage{hypcap}% it must be loaded after hyperref.
% Set up styles of URL: it should be placed after hyperref.
\urlstyle{same}
\addto\captionsenglish{\renewcommand{\contentsname}{Contents:}}

\addto\captionsenglish{\renewcommand{\figurename}{Fig.}}
\addto\captionsenglish{\renewcommand{\tablename}{Table}}
\addto\captionsenglish{\renewcommand{\literalblockname}{Listing}}

\addto\extrasenglish{\def\pageautorefname{page}}

\setcounter{tocdepth}{1}



\title{Match Harry Potter Spells With Their Definition Documentation}
\date{Apr 17, 2017}
\release{1.0}
\author{James Brill}
\newcommand{\sphinxlogo}{}
\renewcommand{\releasename}{Release}
\makeindex

\begin{document}

\maketitle
\sphinxtableofcontents
\phantomsection\label{\detokenize{index::doc}}



\chapter{About}
\label{\detokenize{index:about}}\label{\detokenize{index:welcome-to-match-hp-spells-with-their-definition-s-documentation}}
This Python package has been made as a tool which provides the user with the ability to perform an analysis on two popular vector space models: Google's Word2Vec and Stanford's GloVe. It can either be imported so that other developers can adapt the implemented functions for their own projects, or alternatively, they can run the file hp\_spells.py, see the usage notes for more details.

This documentation briefly describes the main features of the program, as well as notes on how to use the program, from installing the necessary dependencies to the parameters that can be passed.


\chapter{Current Features}
\label{\detokenize{index:current-features}}\begin{itemize}
\item {} 
Can use either Word2Vec or GloVe to generate Harry Potter spell names.

\item {} 
A different list of spells can easily be used by replacing the contents of the file \sphinxquotedblleft{}spells.csv\sphinxquotedblright{} with different contents provided the new content is in the same format.

\item {} 
Provides evaluation against four different metrics:
\begin{enumerate}
\item {} 
Originality

\item {} 
Cosine Similarity

\item {} 
Synoynms

\item {} 
Gibberish Words

\end{enumerate}

\item {} 
Displays the metrics graphically using violin plots and time series plots.

\end{itemize}


\chapter{Usage Notes}
\label{\detokenize{index:usage-notes}}\begin{itemize}
\item {} 
To install the dependencies, navigate to the base level of the directory and use the command \sphinxquotedblleft{}pip install -r requirements.txt\sphinxquotedblright{}.

\item {} 
To execute the program navigate to the \sphinxquotedblleft{}hp\_spells\sphinxquotedblright{} directory, and use the command \sphinxquotedblleft{}Python hp\_spells.py\sphinxquotedblright{} to execute the program with default settings.

\item {} 
There are five different parameters that can be passed to the program.
\begin{enumerate}
\item {} 
--help : This displays the help screen which lists what parameters are available.

\item {} 
--glove : When this parameter is passed, the GloVe vector set is loaded instead of Word2Vec.

\item {} 
--exp EXP : An integer supplied to the program which determines how many times the experiment is repeated.

\item {} 
--comp : This flag is used to run the program in comparison mode.

\item {} 
--verbose : When this parameter is passed, the new spell name is printed out on the screen.

\end{enumerate}

\item {} 
When passing a parameter to the parameter, ensure to use two hyphens.

\end{itemize}


\section{Harry Potter Spell Generator Functions}
\label{\detokenize{code:module-hp_spells}}\label{\detokenize{code:harry-potter-spell-generator-functions}}\label{\detokenize{code::doc}}\index{hp\_spells (module)}\index{calcProb() (in module hp\_spells)}

\begin{fulllineitems}
\phantomsection\label{\detokenize{code:hp_spells.calcProb}}\pysiglinewithargsret{\sphinxcode{hp\_spells.}\sphinxbfcode{calcProb}}{\emph{data}}{}~\begin{quote}

Calculates the probabilities for spells of each type.
\end{quote}
\begin{quote}\begin{description}
\item[{Parameters}] \leavevmode
\sphinxstyleliteralstrong{data} (\sphinxstyleliteralemphasis{}\sphinxstyleliteralemphasis{{[}}\sphinxstyleliteralemphasis{}\sphinxstyleliteralemphasis{{[}}\sphinxstyleliteralemphasis{}\sphinxstyleliteralemphasis{{[}}\sphinxstyleliteralemphasis{str}\sphinxstyleliteralemphasis{, }\sphinxstyleliteralemphasis{str}\sphinxstyleliteralemphasis{{]}}\sphinxstyleliteralemphasis{}\sphinxstyleliteralemphasis{, }\sphinxstyleliteralemphasis{int}\sphinxstyleliteralemphasis{{]}}\sphinxstyleliteralemphasis{..}\sphinxstyleliteralemphasis{{]}}\sphinxstyleliteralemphasis{}) -- List of spell types and origin language with frequency.

\item[{Returns}] \leavevmode
A list of type of spells and their associated probabilities.

\end{description}\end{quote}

\end{fulllineitems}

\index{checkStoredWords() (in module hp\_spells)}

\begin{fulllineitems}
\phantomsection\label{\detokenize{code:hp_spells.checkStoredWords}}\pysiglinewithargsret{\sphinxcode{hp\_spells.}\sphinxbfcode{checkStoredWords}}{\emph{kwords}, \emph{word}}{}~\begin{quote}

This function updates a list of known words with a new word. If the spell type and
\end{quote}

language exists in the list the value is append by 1 otherwise, it is appended to 
the end of the list with a value of 1.
\begin{quote}\begin{description}
\item[{Parameters}] \leavevmode\begin{itemize}
\item {} 
\sphinxstyleliteralstrong{kwords} (\sphinxstyleliteralemphasis{}\sphinxstyleliteralemphasis{{[}}\sphinxstyleliteralemphasis{}\sphinxstyleliteralemphasis{{[}}\sphinxstyleliteralemphasis{}\sphinxstyleliteralemphasis{{[}}\sphinxstyleliteralemphasis{str}\sphinxstyleliteralemphasis{, }\sphinxstyleliteralemphasis{str}\sphinxstyleliteralemphasis{{]}}\sphinxstyleliteralemphasis{}\sphinxstyleliteralemphasis{, }\sphinxstyleliteralemphasis{int}\sphinxstyleliteralemphasis{{]}}\sphinxstyleliteralemphasis{..}\sphinxstyleliteralemphasis{{]}}\sphinxstyleliteralemphasis{}) -- List of spell types and language with associated frequencies.

\item {} 
\sphinxstyleliteralstrong{word} (\sphinxstyleliteralemphasis{str}) -- One being the spell type and the other being the origin language.

\end{itemize}

\item[{Returns}] \leavevmode
the updated list of known words.

\end{description}\end{quote}

\end{fulllineitems}

\index{count\_instances() (in module hp\_spells)}

\begin{fulllineitems}
\phantomsection\label{\detokenize{code:hp_spells.count_instances}}\pysiglinewithargsret{\sphinxcode{hp\_spells.}\sphinxbfcode{count\_instances}}{\emph{fname}}{}~\begin{quote}

Reads supplied file, where it splits it up. Then it appends each word to the data
\end{quote}

set building a list of words and frequencies using checkStoredWords(kwords, word).
\begin{quote}\begin{description}
\item[{Parameters}] \leavevmode
\sphinxstyleliteralstrong{fname} (\sphinxstyleliteralemphasis{str}) -- This is the name of the CSV file in which the spell data is stored.

\item[{Returns}] \leavevmode
returns a list of languages and the probabilities for each one.

\end{description}\end{quote}

\end{fulllineitems}

\index{f() (in module hp\_spells)}

\begin{fulllineitems}
\phantomsection\label{\detokenize{code:hp_spells.f}}\pysiglinewithargsret{\sphinxcode{hp\_spells.}\sphinxbfcode{f}}{\emph{str}}{}
Returns the first two chacters from the string.
\begin{quote}\begin{description}
\item[{Parameters}] \leavevmode
\sphinxstyleliteralstrong{str} (\sphinxstyleliteralemphasis{str}) -- A word that is passed.

\item[{Returns}] \leavevmode
a string that only contains the first two letters.

\end{description}\end{quote}

\end{fulllineitems}

\index{generateScale() (in module hp\_spells)}

\begin{fulllineitems}
\phantomsection\label{\detokenize{code:hp_spells.generateScale}}\pysiglinewithargsret{\sphinxcode{hp\_spells.}\sphinxbfcode{generateScale}}{\emph{data}}{}~\begin{quote}

This stacks the probabilities of spells so that each spell has a boundary in which
\end{quote}

it a spell can be selected over another.
\begin{quote}\begin{description}
\item[{Parameters}] \leavevmode
\sphinxstyleliteralstrong{data} (\sphinxstyleliteralemphasis{}\sphinxstyleliteralemphasis{{[}}\sphinxstyleliteralemphasis{}\sphinxstyleliteralemphasis{{[}}\sphinxstyleliteralemphasis{}\sphinxstyleliteralemphasis{{[}}\sphinxstyleliteralemphasis{str}\sphinxstyleliteralemphasis{,}\sphinxstyleliteralemphasis{str}\sphinxstyleliteralemphasis{{]}}\sphinxstyleliteralemphasis{}\sphinxstyleliteralemphasis{,}\sphinxstyleliteralemphasis{int}\sphinxstyleliteralemphasis{,}\sphinxstyleliteralemphasis{float}\sphinxstyleliteralemphasis{{]}}\sphinxstyleliteralemphasis{..}\sphinxstyleliteralemphasis{{]}}\sphinxstyleliteralemphasis{}) -- list of spell names and their associated frequencies and probabilities.

\item[{Returns}] \leavevmode
a list of spells and the value between 0-1 in which that name will be selected.

\end{description}\end{quote}

\end{fulllineitems}

\index{generateSpell() (in module hp\_spells)}

\begin{fulllineitems}
\phantomsection\label{\detokenize{code:hp_spells.generateSpell}}\pysiglinewithargsret{\sphinxcode{hp\_spells.}\sphinxbfcode{generateSpell}}{\emph{sentence}, \emph{model}, \emph{oword}}{}
Generates a Spell from a sentence.
\begin{quote}\begin{description}
\item[{Parameters}] \leavevmode\begin{itemize}
\item {} 
\sphinxstyleliteralstrong{sentence} (\sphinxstyleliteralemphasis{str}) -- string which is the definition of the spell you want to create.

\item {} 
\sphinxstyleliteralstrong{model} -- loaded vector orepresentation of words.
:type model: data file loaded.

\end{itemize}

\item[{Returns}] \leavevmode
list containing the spell and the spell type.

\end{description}\end{quote}

\end{fulllineitems}

\index{getSpellType() (in module hp\_spells)}

\begin{fulllineitems}
\phantomsection\label{\detokenize{code:hp_spells.getSpellType}}\pysiglinewithargsret{\sphinxcode{hp\_spells.}\sphinxbfcode{getSpellType}}{\emph{scale}, \emph{rndNum}}{}~\begin{quote}

Selects a spell according to the random number passed.
\end{quote}
\begin{quote}\begin{description}
\item[{Parameters}] \leavevmode\begin{itemize}
\item {} 
\sphinxstyleliteralstrong{scale} (\sphinxstyleliteralemphasis{}\sphinxstyleliteralemphasis{{[}}\sphinxstyleliteralemphasis{}\sphinxstyleliteralemphasis{(}\sphinxstyleliteralemphasis{str}\sphinxstyleliteralemphasis{,}\sphinxstyleliteralemphasis{str}\sphinxstyleliteralemphasis{,}\sphinxstyleliteralemphasis{float}\sphinxstyleliteralemphasis{)}\sphinxstyleliteralemphasis{.}\sphinxstyleliteralemphasis{{]}}\sphinxstyleliteralemphasis{}) -- A list of tuples which contains the probability associated with each spell and type.

\item {} 
\sphinxstyleliteralstrong{rndNum} (\sphinxstyleliteralemphasis{float}) -- The random number used to select a spell type.

\end{itemize}

\item[{Returns}] \leavevmode
A string which is the spell type.

\end{description}\end{quote}

\end{fulllineitems}

\index{is\_synonym() (in module hp\_spells)}

\begin{fulllineitems}
\phantomsection\label{\detokenize{code:hp_spells.is_synonym}}\pysiglinewithargsret{\sphinxcode{hp\_spells.}\sphinxbfcode{is\_synonym}}{\emph{n\_word}, \emph{o\_word}}{}
This function uses a combination of NLTK's wordnet to 
list all synonyms for a word and to check if a new word is a synonym.
\begin{quote}\begin{description}
\item[{Parameters}] \leavevmode\begin{itemize}
\item {} 
\sphinxstyleliteralstrong{n\_word} (\sphinxstyleliteralemphasis{str}) -- The new word generated.

\item {} 
\sphinxstyleliteralstrong{o\_word} (\sphinxstyleliteralemphasis{str}) -- The original word in the definition.

\end{itemize}

\item[{Returns}] \leavevmode
Returns a boolean indicating whether n\_word is a synonym of o\_word.

\end{description}\end{quote}

\end{fulllineitems}

\index{is\_valid() (in module hp\_spells)}

\begin{fulllineitems}
\phantomsection\label{\detokenize{code:hp_spells.is_valid}}\pysiglinewithargsret{\sphinxcode{hp\_spells.}\sphinxbfcode{is\_valid}}{\emph{string}}{}
check to see whether a word consists of alpha characters.
\begin{quote}\begin{description}
\item[{Parameters}] \leavevmode
\sphinxstyleliteralstrong{string} (\sphinxstyleliteralemphasis{str}) -- The string to be checked.

\item[{Returns}] \leavevmode
Boolean value.

\end{description}\end{quote}

\end{fulllineitems}

\index{langCode() (in module hp\_spells)}

\begin{fulllineitems}
\phantomsection\label{\detokenize{code:hp_spells.langCode}}\pysiglinewithargsret{\sphinxcode{hp\_spells.}\sphinxbfcode{langCode}}{\emph{language}}{}
Converts a language name into a language code for the translator.
\begin{quote}\begin{description}
\item[{Parameters}] \leavevmode
\sphinxstyleliteralstrong{language} (\sphinxstyleliteralemphasis{tr}) -- Full name of the language, for example latin.

\item[{Returns}] \leavevmode
The string code for the language.

\end{description}\end{quote}

\end{fulllineitems}

\index{load\_vectors() (in module hp\_spells)}

\begin{fulllineitems}
\phantomsection\label{\detokenize{code:hp_spells.load_vectors}}\pysiglinewithargsret{\sphinxcode{hp\_spells.}\sphinxbfcode{load\_vectors}}{\emph{path}, \emph{is\_binary}}{}
This loads the vectors supplied by the path.
\begin{quote}\begin{description}
\item[{Parameters}] \leavevmode\begin{itemize}
\item {} 
\sphinxstyleliteralstrong{path} (\sphinxstyleliteralemphasis{str}) -- The path to the vector file

\item {} 
\sphinxstyleliteralstrong{is\_binary} (\sphinxstyleliteralemphasis{boolean}) -- states whether file is a binary file.

\end{itemize}

\item[{Returns}] \leavevmode
The loaded model.

\end{description}\end{quote}

\end{fulllineitems}

\index{pigLatin() (in module hp\_spells)}

\begin{fulllineitems}
\phantomsection\label{\detokenize{code:hp_spells.pigLatin}}\pysiglinewithargsret{\sphinxcode{hp\_spells.}\sphinxbfcode{pigLatin}}{\emph{source}}{}~\begin{quote}

Takes a source string and converts it from english to pig latin.
\end{quote}
\begin{quote}\begin{description}
\item[{Parameters}] \leavevmode
\sphinxstyleliteralstrong{source} (\sphinxstyleliteralemphasis{str}) -- Takes string of english words and changes it into pig latin.

\item[{Returns}] \leavevmode
a string containing pig latin words.

\end{description}\end{quote}

\end{fulllineitems}

\index{run\_experiment() (in module hp\_spells)}

\begin{fulllineitems}
\phantomsection\label{\detokenize{code:hp_spells.run_experiment}}\pysiglinewithargsret{\sphinxcode{hp\_spells.}\sphinxbfcode{run\_experiment}}{\emph{model}, \emph{num\_experiments}}{}
This function runs the experiments with the paramters set. 
It then returns all the necessary data for processing and output.
\begin{quote}\begin{description}
\item[{Parameters}] \leavevmode\begin{itemize}
\item {} 
\sphinxstyleliteralstrong{model} (\sphinxstyleliteralemphasis{The loaded vector object}) -- The vectors loaded.

\item {} 
\sphinxstyleliteralstrong{num\_experiments} (\sphinxstyleliteralemphasis{int}) -- The number of experiments to run.

\end{itemize}

\item[{Returns}] \leavevmode
A list of averages scores, one entry per experiment.

\item[{Returns}] \leavevmode
A list of the average number of synonyms produced, one  entry per experiment.

\item[{Returns}] \leavevmode
The average score across the experiments.

\item[{Returns}] \leavevmode
A list of average cosine similarity scores, one entry per experiment.

\item[{Returns}] \leavevmode
The number of experiments.

\item[{Returns}] \leavevmode
A list containing the number of bogus words produced, one entry per expeirment.

\item[{Returns}] \leavevmode
A list containing lists with each sublist containing the scores produced for that definition length.

\item[{Returns}] \leavevmode
A list containing list with each sublist containing number of bogus words produced for that definition length.

\end{description}\end{quote}

\end{fulllineitems}

\index{sentenceToWord() (in module hp\_spells)}

\begin{fulllineitems}
\phantomsection\label{\detokenize{code:hp_spells.sentenceToWord}}\pysiglinewithargsret{\sphinxcode{hp\_spells.}\sphinxbfcode{sentenceToWord}}{\emph{sentence}, \emph{model}, \emph{oword}}{}
Takes a string and converts it into a vector. Then from that it picks a similar word that doesn't contain an underscore.
\begin{quote}\begin{description}
\item[{Parameters}] \leavevmode
\sphinxstyleliteralstrong{sentence} (\sphinxstyleliteralemphasis{str}) -- A string which contains a sentence to be converted into one word.

\item[{Returns}] \leavevmode
A string containing a similar word.

\end{description}\end{quote}

\end{fulllineitems}

\index{totalSpells() (in module hp\_spells)}

\begin{fulllineitems}
\phantomsection\label{\detokenize{code:hp_spells.totalSpells}}\pysiglinewithargsret{\sphinxcode{hp\_spells.}\sphinxbfcode{totalSpells}}{\emph{data}}{}~\begin{quote}

Counts the number of spells in the dataset.
\end{quote}
\begin{quote}\begin{description}
\item[{Parameters}] \leavevmode
\sphinxstyleliteralstrong{data} (\sphinxstyleliteralemphasis{}\sphinxstyleliteralemphasis{{[}}\sphinxstyleliteralemphasis{}\sphinxstyleliteralemphasis{{[}}\sphinxstyleliteralemphasis{}\sphinxstyleliteralemphasis{{[}}\sphinxstyleliteralemphasis{str}\sphinxstyleliteralemphasis{,}\sphinxstyleliteralemphasis{str}\sphinxstyleliteralemphasis{{]}}\sphinxstyleliteralemphasis{}\sphinxstyleliteralemphasis{, }\sphinxstyleliteralemphasis{int}\sphinxstyleliteralemphasis{{]}}\sphinxstyleliteralemphasis{..}\sphinxstyleliteralemphasis{{]}}\sphinxstyleliteralemphasis{}) -- List of spell types and origin language with frequency.

\item[{Returns}] \leavevmode
an integer value of total number of spells.

\end{description}\end{quote}

\end{fulllineitems}

\index{translate2() (in module hp\_spells)}

\begin{fulllineitems}
\phantomsection\label{\detokenize{code:hp_spells.translate2}}\pysiglinewithargsret{\sphinxcode{hp\_spells.}\sphinxbfcode{translate2}}{\emph{word}, \emph{lang}}{}~\begin{quote}

Translates a word to a target language.
\end{quote}
\begin{quote}\begin{description}
\item[{Parameters}] \leavevmode\begin{itemize}
\item {} 
\sphinxstyleliteralstrong{word} (\sphinxstyleliteralemphasis{str}) -- The word you want to convert.

\item {} 
\sphinxstyleliteralstrong{lang} (\sphinxstyleliteralemphasis{str}) -- the lang code of the language you want to convert to.

\end{itemize}

\item[{Returns}] \leavevmode
a string containing the translated word in the latin alphabet.

\end{description}\end{quote}

\end{fulllineitems}



\chapter{Indices and tables}
\label{\detokenize{index:indices-and-tables}}\begin{itemize}
\item {} 
\DUrole{xref,std,std-ref}{genindex}

\item {} 
\DUrole{xref,std,std-ref}{modindex}

\item {} 
\DUrole{xref,std,std-ref}{search}

\end{itemize}


\renewcommand{\indexname}{Python Module Index}
\begin{sphinxtheindex}
\def\bigletter#1{{\Large\sffamily#1}\nopagebreak\vspace{1mm}}
\bigletter{h}
\item {\sphinxstyleindexentry{hp\_spells}}\sphinxstyleindexpageref{code:\detokenize{module-hp_spells}}
\end{sphinxtheindex}

\renewcommand{\indexname}{Index}
\printindex
\end{document}